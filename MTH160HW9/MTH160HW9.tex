\documentclass{ximera}

\author{Anna Davis} \title{MTH 160 Homework 9} 

\begin{document}

\begin{abstract}

\end{abstract}
\maketitle
 \textit{Certificate due: 11/3/2021 at 11:59 p.m.}
 \section{Lecture 20}
 
  \begin{problem}\label{prob:160hom8prob1} 
Use properties of logarithms to simplify without the use of a calculator.
\begin{enumerate}
    \item $e^{3\ln x}=\answer{x^3}$
    \item $\ln\left(\ln e^2\right)=\ln\answer{2}$
    \item $\log 2 +\log 50=\log \answer{100}=\answer{2}$
\end{enumerate}
\end{problem}

\begin{problem}\label{prob:160hom8prob8}
 Simplify as much as possible.
 \begin{enumerate}
     \item $\log_b \frac{1}{b}=\answer{-1}$
     \item $2\log 2+\log\frac{1}{4}=\answer{0}$
     \item $10^{\log a}=\answer{a}$
     \item $\ln\left(\log_b b\right)=\answer{0}$
 \end{enumerate}
 \end{problem}

\begin{problem}\label{prob:160hom8prob2}
Use change of base formula to evaluate. 

$$\log_3 12=\frac{\log\answer{12}}{\log\answer{3}}=\answer[tolerance=0.01]{2.26}$$
\end{problem}

\begin{problem}\label{prob:160hom8prob3}
Write the expression as a single logarithm.
$$\log x-3\log y+\frac{1}{2}\log z=\log\left(\frac{\answer{x\sqrt{z}}}{\answer{y^3}}\right)$$
\end{problem}

\begin{problem}\label{prob:160hom8prob10}
 Write as a single logarithm.
 
 $$\frac{1}{2}\log m+4\log n+5\log p=\answer{\log(\sqrt{m}n^4p^5)}$$
 \end{problem}
 
 \begin{problem}\label{prob:160hom8prob11}
 Write as a sum/difference of logarithms.
 
 $$\ln \frac{\sqrt{z}}{b^2}=\answer{0.5\ln z-2\ln b}$$
 \end{problem}

 
 
  \section{Lecture 21}
 
 \begin{problem}\label{prob:160hom8prob4}
Follow the indicated steps to solve for $x$.
$$300=150(1.12)^{24x}$$
Divide both sides by 150.
$$\answer{2}=1.12^{24x}$$
Take the natural log of both sides and bring the exponent to the front.
$$\answer[tolerance=0.01]{0.693}=\answer{24x}\ln\answer{1.12}$$
$$x=\answer[tolerance=0.01]{0.255}$$
\end{problem}

\begin{problem}\label{prob:160hom8prob5}
Solve for $x$.
$$3^{x-1}=9^{2-x}$$
$$x=\answer[tolerance=0.01]{\frac{5}{3}}$$
\end{problem}

\begin{problem}\label{prob:160hom8prob6}
Follow the indicated steps to solve for $x$.
$$\log (2x+3)=-1$$
Rewrite as an exponential expression.
$$10^{\answer{-1}}=\answer{2x+3}$$
$$x=\answer{-1.45}$$
\end{problem}

\begin{problem}\label{prob:160hom8prob7}
Solve for $x$.
$$450=700e^{-0.27x}$$
$$x=\answer[tolerance=0.01]{1.636}$$
\end{problem}

\begin{problem}\label{prob:160hom8prob9}
 Solve each equation for $x$.
 \begin{enumerate}
     \item $$16^{x-1}=2^x$$
     $$x=\answer[tolerance=0.03]{\frac{4}{3}}$$
     \item 
     $$100e^{3x}=120$$
     $$x=\answer[tolerance=0.001]{0.06}$$
     \item 
     $$10\log (2x)=20$$
     $$x=\answer{50}$$
 \end{enumerate}
 \end{problem}
 
 \section{Lecture 22}
 
  \begin{problem}\label{prob:160hom9prob1} 
Suppose \$3000 is deposited into an account which pays 2.5\% annual interest compounded daily.  
 \begin{enumerate}
 \item Fill in the given information:
 $$P=\$\answer{3000},\quad r=\answer{0.025},\quad n=\answer{365}$$
 \item How long will it take for the investment to grow to \$3500?
 Round your answer to one decimal place.
 $$t=\answer[tolerance=0.1]{6.2}$$
 \end{enumerate}
 \end{problem}
 
 \begin{problem}\label{prob:160hom9prob2} 
 Suppose a radioactive isotope decays according to the formula $A(t)=Pe^{-0.032t}$, where $t$ is measured in hours.  
 \begin{enumerate}
     \item 
If we start with 200 grams of the isotope, the amount left after 2 hours is: $\answer[tolerance=0.01]{187.6}$ grams.
\item The half-life of the isotope is: $\answer[tolerance=0.1]{21.66}$ hours.
\end{enumerate}
\end{problem}

\begin{problem}\label{prob:160hom9prob3}
 Suppose that the temperature of a pie taken out of the oven can be modeled by $T(t)=70+280e^{-0.7t}$, where $t$ is measured in hours.
 \begin{enumerate}
     \item The pie was baked at the temperature of $\answer{350}$ degrees.
     \item How long will it take for the pie to cool to 90 degrees?  Round your answers to two decimal places.
     $$t=\answer[tolerance=0.1]{3.77}$$
 \end{enumerate}
 \end{problem}
 
 \end{document}