\documentclass{ximera}

\author{Anna Davis} \title{MTH 140 Lab 3: Impact of Assessment Strategy on Grades} 

\begin{document}

\begin{abstract}

\end{abstract}
\maketitle
 \textit{Certificate due: 12/4/2020 at 11:59 p.m.}
 \section{Independent Samples}
\begin{problem}\label{prob:140lab3prob1}
Enter the information for the sample of final grades from a pre-COVID class.
Round to two decimal places.
$$\overline{x}_1=\answer{82.42};\quad s_1=\answer{12.48};\quad n_1=\answer{105}$$

Enter the information for the sample of final grades from a post-COVID class.
Round to two decimal places.
$$\overline{x}_2=\answer{82.06};\quad s_2=\answer{14.82};\quad n_2=\answer{71}$$

Use the Desmos template to find the test statistic and degrees of freedom.  Use the rounded numbers from the previous step.

Test statistic (round to four decimal places): $t=\answer{0.1683}$

Degrees of freedom (round to two decimal places): df$=\answer{132.69}$

Find the $p$-value (for a two-tail hypothesis test) using Geogebra (to four decimal places).

\begin{center}  
\geogebra{emwxhga2}{800}{600}  
\end{center}

$$p=\answer{0.8666}$$

Find the $p$-value using the spreadsheet function "=T.TEST".  Round to four decimal places.

$$p=\answer{0.8681}$$
\begin{warning}
Observe that the two $p$-values are slightly different due to rounding.  It does not make much of a difference, in this case.  You can use either one to make the decision.
\end{warning}
\end{problem}



\section{Paired Samples}

\begin{problem}\label{prob:140lab3prob2}
Enter the information for the paired samples.  Round to two decimal places.

$$\overline{d}=\answer{6.22};\quad s_d=\answer{11.8};\quad n=\answer{49}$$

Find the test statistic.  Round to two decimal places.
$$t=\answer{3.69}$$

Find the $p$-value using Geogebra (to four decimal places).

\begin{center}  
\geogebra{emwxhga2}{800}{600}  
\end{center}

$$p=\answer{0.0003}$$

Find the $p$-value using the spreadsheet function "=T.TEST".  Round to FIVE decimal places.
$$p=\answer{0.00028}$$
\end{problem}
\end{document}