\documentclass{ximera}


\author{Anna Davis} \title{MTH 160 Test 3} 

\begin{document}

\begin{abstract}

\end{abstract}
\maketitle
 \textit{You have 50 minutes to complete this test.  Each answer is worth 1 point.}
 \begin{problem}\label{prob:160test3prob1}
 Complete the statement.
 \begin{enumerate}
     \item 
 $\log_b a=p$ means $\answer{b^p}=\answer{a}$
 \item
 $\log x=y$ means $\answer{10^y}=\answer{x}$
 \end{enumerate}
 \end{problem}
 
 \begin{problem}\label{prob:160test3prob2}
 Simplify as much as possible.
 \begin{enumerate}
     \item $\log_b \frac{1}{b}=\answer{-1}$
     \item $2\log 2+\log\frac{1}{4}=\answer{0}$
     \item $10^{\log a}=\answer{a}$
     \item $\ln\left(\log_b b\right)=\answer{0}$
 \end{enumerate}
 \end{problem}
 
 \begin{problem}\label{prob:160test3prob3}
 Solve each equation for $x$.
 \begin{enumerate}
     \item $$16^{x-1}=2^x$$
     $$x=\answer[tolerance=0.03]{\frac{4}{3}}$$
     \item 
     $$100e^{3x}=120$$
     $$x=\answer[tolerance=0.001]{0.06}$$
     \item 
     $$10\log (2x)=20$$
     $$x=\answer{50}$$
 \end{enumerate}
 \end{problem}
 
 \begin{problem}\label{prob:160test3prob4}
 Write as a single logarithm.
 
 $$\frac{1}{2}\log m+4\log n+5\log p=\answer{\sqrt{m}n^4p^5}$$
 \end{problem}
 
 \begin{problem}\label{prob:160test3prob5}
 Write as a sum/difference of logarithms.
 
 $$\ln \frac{\sqrt{z}}{b^2}=\answer{0.5\ln z-2\ln b}$$
 \end{problem}
 
 \begin{problem}\label{prob:160test3prob6}
 Suppose \$3000 is deposited into an account which pays 2.5\% annual interest compounded daily.  
 \begin{enumerate}
 \item Fill in the given information:
 $$P=\$\answer{3000},\quad r=\answer{0.025},\quad n=\answer{365}$$
 \item How long will it take for the investment to grow to \$3500?
 Round your answer to one decimal place.
 $$t=\answer[tolerance=0.1]{6.2}$$
 \end{enumerate}
 \end{problem}
 
 \begin{problem}\label{prob:160test3prob7}
 Suppose a radioactive isotope decays according to the formula $A(t)=Pe^{-0.032t}$, where $t$ is measured in hours.  
 \begin{enumerate}
     \item 
If we start with 200 grams of the isotope, the amount left after 2 hours is: $\answer[tolerance=0.01]{187.6}$ grams.
\item The half-life of the isotope is: $\answer[tolerance=0.1]{21.66}$ hours.
\end{enumerate}
 \end{problem}
 
 \begin{problem}\label{prob:160test3prob8}
 Suppose that the temperature of a pie taken out of the oven can be modeled by $T(t)=70+280e^{-0.7t}$, where $t$ is measured in hours.
 \begin{enumerate}
     \item The pie was baked at the temperature of $\answer{350}$ degrees.
     \item How long will it take for the pie to cool to 90 degrees?  Round your answers to two decimal places.
     $$t=\answer[tolerance=0.01]{3.77}$$
 \end{enumerate}
 \end{problem}
 
 \end{document}