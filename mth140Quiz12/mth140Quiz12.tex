\documentclass{ximera}

\author{Anna Davis} \title{MTH 140 Quiz 12} 

\begin{document}

\begin{abstract}

\end{abstract}
\maketitle
 \textit{Soft Deadline: April 20, 2020}
\begin{problem}\label{prob:140quiz12prob1}
Suppose a student advocacy group is trying to determine whether there is a difference between the average textbook cost for college freshmen and college seniors.  The group surveys eight randomly selected freshmen and eleven seniors.  The results are listed below.  Perform hypothesis testing at 0.05 level of significance. (You may assume that textbook costs are approximately normally distributed.)

Freshmen: \$155, \$280, \$98, \$163, \$178, \$210, \$173, \$169

Seniors: \$145, \$180, \$197, \$238, \$215, \$110, \$153, \$116, \$203, \$54, \$105  

Summarize the information for college freshmen.
$$\overline{x}_f=\answer[tolerance=0.01]{178.25},\quad s_f=\answer[tolerance=0.01]{51.67},\quad n_f=\answer{8}$$

Summarize the information for college seniors.
$$\overline{x}_s=\answer[tolerance=0.01]{156},\quad s_s=\answer[tolerance=0.01]{56.14},\quad n_s=\answer{11}$$

Null Hypothesis:

$H_0:\quad \mu_f-\mu_s$ \wordChoice{\choice{$\leq0$}, \choice[correct]{$=0$}, \choice{$\geq 0$}}

Alternative Hypothesis:

$H_a:\quad \mu_f-\mu_s$ \wordChoice{\choice{$<0$}, \choice[correct]{$\neq 0$}, \choice{$>0$}}

Find the $t$-score
$$t=\answer[tolerance=0.01]{0.893}$$
Find degrees of freedom
$$\mbox{df}=\answer[tolerance=0.01]{15.95}$$
Find the $p$-value
\begin{center}  
\geogebra{emwxhga2}{800}{600}  
\end{center}
$$p=\answer[tolerance=0.01]{0.3852}$$
Decision:

\begin{multipleChoice} 
\choice{$p$-value is less than 0.05.  We reject the null hypothesis.}  
\choice{$p$-value is less than 0.05.  We do not reject the null hypothesis.} 
\choice{$p$-value is greater than 0.05.  We reject the null hypothesis.} 
\choice[correct]{$p$-value is greater than 0.05.  We do not reject the null hypothesis.}  
\end{multipleChoice}  

Conclusion:

\begin{multipleChoice} 
\choice{At the 0.05 level of significance, we have proved the null hypothesis.}  
\choice{At the 0.05 level of significance, there is not enough evidence to make any conclusions.} 
\choice{At the 0.05 level of significance, there is enough evidence to conclude that the mean textbook cost for freshmen is higher than the mean textbook cost for seniors.}  
\choice[correct]{At the 0.05 level of significance, there is not enough evidence to conclude that the mean textbook cost for freshmen is different from the mean textbook cost for seniors.} 
\end{multipleChoice} 
\end{problem}

\begin{problem}\label{prob:140quiz12prob2}
Suppose a consumer advocacy group believes that brand $B$ tires last longer than brand $A$ tires.  The group surveys ten randomly selected tire owners for each of the two brands.  The longevity results (in thousands of miles) are listed below.  Suppose the standard deviation for longevity for brand $A$ is known to be $7500$ miles, and the standard deviation for brand $B$ is known to be $8100$.  Perform hypothesis testing at 0.05 level of significance. (You may assume that tire longevity is approximately normally distributed.)

Brand $A$ (in thousands of miles): $$61.5,\quad 59.3,\quad 52.8,\quad 64.1,\quad 60.4,\quad 51.4,\quad 61.3,\quad 67.8,\quad 55.7,\quad 61.1$$

Brand $B$ (in thousands of miles): $$60.3,\quad 60.1, \quad 56.7,\quad 69.5, \quad 67.4,\quad 57.1,\quad 69.8,\quad 62.4,\quad 57.4,\quad 68.1$$

Summarize the information for brand $A$ (in thousands of miles).
$$\overline{x}_A=\answer[tolerance=0.01]{59.54},\quad \sigma_A=\answer[tolerance=0.01]{7.5},\quad n_A=\answer{10}$$

Summarize the information for brand $B$ (in thousands of miles).
$$\overline{x}_B=\answer[tolerance=0.01]{62.88},\quad \sigma_B=\answer[tolerance=0.01]{8.1},\quad n_B=\answer{10}$$

Null Hypothesis:

$H_0:\quad \mu_B-\mu_A=0$ 

Alternative Hypothesis:

$H_a:\quad \mu_B-\mu_A$ \wordChoice{\choice{$<0$}, \choice{$\neq 0$}, \choice[correct]{$>0$}}

Find the $z$-score
$$z=\answer[tolerance=0.01]{0.9568}$$
Find the $p$-value
\begin{center}  
\geogebra{emwxhga2}{800}{600}  
\end{center}
$$p=\answer[tolerance=0.01]{0.1693}$$
Decision:

\begin{multipleChoice} 
\choice{$p$-value is less than 0.05.  We reject the null hypothesis.}  
\choice{$p$-value is less than 0.05.  We do not reject the null hypothesis.} 
\choice{$p$-value is greater than 0.05.  We reject the null hypothesis.} 
\choice[correct]{$p$-value is greater than 0.05.  We do not reject the null hypothesis.}  
\end{multipleChoice}  

Conclusion:

\begin{multipleChoice} 
\choice{At the 0.05 level of significance, we have proved the null hypothesis.}  \choice{At the 0.05 level of significance, there is not enough evidence to make any conclusions.} 
\choice{At the 0.05 level of significance, there is enough evidence to conclude that the mean tire longevity for brand $B$ is greater than the mean longevity for brand $A$.}  
\choice[correct]{At the 0.05 level of significance, there is not enough evidence to conclude that the mean tire longevity for brand $B$ is greater than the mean longevity for brand $A$.} 
\end{multipleChoice} 
\end{problem}

\end{document} 